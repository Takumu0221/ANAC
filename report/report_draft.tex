% sample.tex
\documentclass[uplatex, 10pt, a4j]{jsarticle}
\usepackage{authblk}
\usepackage{amsmath}

\begin{document}
% 本文
\title{Gentle}
\author{Takumu Shimizu}
\affil{Tokyo University of Agriculture and Technology}
\date{}
\maketitle

\section{Negotiation management}
このエージェントは,それぞれの相手に対して独立して交渉を行う.
\subsection{offering strategy}
このエージェントは,いくつかの指標をもとにofferの価格($p^{offer}$)および取引量($q^{offer}$)を決定する.特にutility functionは設定していない.
\begin{equation*}
    \begin{split}
        % offer price
        p^{offer}&=\left\{\begin{array}{ll}
            \frac{p^{max}+p^{min}}{2} \times \left(1+b^{is\_selling}\times s_{offer}\right) & \mathrm{if} \; n^{contract}_{accept}+n^{contract}_{offer}=0                                  \\ [3mm]
            p^{worst\_contract}                                                             & \mathrm{if} \; n^{contract}_{accept} \leq 1 \; \mathrm{and} \; p^{worst\_contract} \geq \tau \\ [1mm]
                                                                                            & \mathrm{or} \; n^{contract}_{accept}<1 \; \mathrm{and} \; n^{contract}_{offer} \geq 1        \\ [3mm]
            p^{worst\_contract} \times \left(1 + b^{is\_selling} \times 0.1\right)          & \mathrm{if} \; n^{contract}_{accept} \geq 1 \; \mathrm{and} \; p^{worst\_contract} < \tau    \\
        \end{array}\right. \\
        % その他変数の説明
        b^{is\_selling}&=\left\{\begin{array}{ll}
            1  & \mathrm{if \; selling} \\
            -1 & \mathrm{if \; buying}
        \end{array}\right. \\
        s_{offer} &= 0.2-0.5 \times \min{\left(\frac{t}{0.3}, 1\right)} \\
        \tau &= \left\{\begin{array}{ll}
            p^{max} - 19^{-0.2} \times \left(p^{max} - p^{min}\right) & \mathrm{if \; selling} \\
            p^{min} + 19^{-0.2} \times \left(p^{max} - p^{min}\right) & \mathrm{if \; buying}  \\
        \end{array}\right. \\
    \end{split}
\end{equation*}
\begin{equation*}
    \begin{split}
        p^{max} \; / \; p^{min} &: \; \text{その交渉でのunit priceの最大値および最小値} \\
        p^{worst\_contract} &: \; \text{交渉相手との合意価格のなかで最も自分にとって悪いもの} \\
        n^{contract}_{offer} \; / \; n^{contract}_{accept} &: \; \text{自分のoffer/acceptによる交渉相手との合意の数} \\
        i \; / \; j \; / \; t &: \; \text{simulation step/negotiation step/negotiation time} \\
    \end{split}
\end{equation*}

\subsection{acceptance strategy}
このエージェントは,いくつかの指標をもとに受け入れるunit price($p^accept$)を決定する.
\begin{equation*}
    \begin{split}
        % accept price
        p^{accept} &= \left\{\begin{array}{ll}
            \max \left(p^{best\_selling}_{i}, p^{best_contract} \times \left(1+s_{accept}\right) \right) & \mathrm{if \; selling} \\ [3mm]
            \min \left(p^{best\_buying}_{i}, p^{best_contract} \times \left(1-s_{accept}\right) \right)  & \mathrm{if \; buying}  \\
        \end{array}\right. \\
        s_{accept} &= \left\{\begin{array}{ll}
            0.2 & \mathrm{if} \; r \geq 3 \\
            0   & \mathrm{if} \; r < 3    \\
        \end{array}\right. \\
        r &= \frac{p^{opp\_offer}_{j} - p^{opp\_offer}_{j-1}}{p^{opp\_offer}_{j-1} - p^{opp\_offer}_{j-2}} \\
    \end{split}
\end{equation*}
\begin{equation*}
    \begin{split}
        p^{best\_contract} &: \; \text{交渉相手との合意価格の中で最も自分にとって良いもの} \\
    \end{split}
\end{equation*}

\section{Simultaneous negotiations coordination:}

\section{Risk management}

\section{Evaluation}
\end{document}
