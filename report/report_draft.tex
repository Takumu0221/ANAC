% sample.tex
\documentclass[uplatex, 10pt, a4j]{jsarticle}
\usepackage{authblk}
\usepackage{amsmath}
\usepackage{amssymb}
\usepackage{amsfonts}
\usepackage{latexsym}
\usepackage[top=20truemm,bottom=20truemm,left=15truemm,right=15truemm]{geometry}
\usepackage{enumerate}

\begin{document}
% 本文
\title{Gentle}
\author{Takumu Shimizu}
\affil{Tokyo University of Agriculture and Technology}
\date{\vspace{-15mm}}

\maketitle

\section{\textrm{Negotiation Management}}
Gentleは,サンプルコードの一つであるAdaptive Aegntを継承しており,それぞれの相手に対して独立に交渉を行う.
また,交渉の選択は継承元のクラスと同一であり,自分のneed($q^{need}$)が満たされるまで全ての相手との交渉を継続する.$q^{need}$は外的取引量からその日の合意した取引量を引いたものである.
特にutility functionは設定しておらず,主に価格を重視して判断する.

\subsection{\textrm{Offering Strategy}}
Gentleは,いくつかの指標をもとにofferの価格($p^{offer}$)および取引量($q^{offer}$)を決定する.
その際,これらの記号 $\trianglerighteq, \vartriangleleft$ を,エージェントにとって良いか悪いかを表す大小関係として定義する.
この時,売り手エージェントであれば $5\vartriangleleft 10$は真であり,買い手エージェントであれば $5 \trianglerighteq 10$も真となる.
\begin{equation*}
    \begin{split}
        % offer price
        p^{offer}&=\left\{\begin{array}{ll}
            \dfrac{p^{max}+p^{min}}{2} \times \left(1+b^{is\_selling}\times s_{offer}\right) & \mathrm{if} \; n^{contract}_{accept}+n^{contract}_{offer}=0                                               \\ [3mm]
            p^{worst\_contract}                                                              & \mathrm{if} \; n^{contract}_{accept} \geq 1 \; \mathrm{and} \; p^{worst\_contract} \trianglerighteq  \tau \\ [1mm]
                                                                                             & \mathrm{or} \; n^{contract}_{accept}<1 \; \mathrm{and} \; n^{contract}_{offer} \geq 1                     \\ [3mm]
            p^{worst\_contract} \times \left(1 + b^{is\_selling} \times 0.1\right)           & \mathrm{if} \; n^{contract}_{accept} \geq 1 \; \mathrm{and} \; p^{worst\_contract} \vartriangleleft \tau  \\
        \end{array}\right. \\
        q^{offer} &= q^{need} \\
        % その他変数の説明
        b^{is\_selling}&=\left\{\begin{array}{ll}
            1  & \mathrm{if \; selling} \\
            -1 & \mathrm{if \; buying}  \\
        \end{array}\right. \\
        s_{offer} &= 0.2-0.5 \times \min{\left(\dfrac{t^{nego}}{0.3}, 1\right)} \\
        \tau &= \left\{\begin{array}{ll}
            p^{max} - 19^{-0.2} \times \left(p^{max} - p^{min}\right) & \mathrm{if \; selling} \\
            p^{min} + 19^{-0.2} \times \left(p^{max} - p^{min}\right) & \mathrm{if \; buying}  \\
        \end{array}\right. \\
    \end{split}
\end{equation*}
\begin{equation*}
    \begin{split}
        p^{max} \; / \; p^{min} &: \; \text{その交渉でのunit priceの最大値および最小値} \\
        p^{worst\_contract} &: \; \text{交渉相手との合意価格のなかで自分にとって最も悪いもの} \\
        n^{contract}_{offer} \; / \; n^{contract}_{accept} &: \; \text{自分のoffer/acceptによる交渉相手との合意の数} \\
        i \; / \; j \; / \; t^{nego} &: \; \text{simulation step/negotiation step/negotiation time} \\
    \end{split}
\end{equation*}

\subsection{\textrm{Acceptance Strategy}}
Gentleは,いくつかの指標をもとに受け入れるunit price($p^{accept}$)を決定する.取引量については考慮しない.
\begin{equation*}
    \begin{split}
        % accept price
        p^{accept} &= \left\{\begin{array}{ll}
            \max \left(p^{best\_selling}_{i}, p^{best\_contract} \times \left(1+s_{accept}\right) \right) & \mathrm{if \; selling} \\ [3mm]
            \min \left(p^{best\_buying}_{i}, p^{best\_contract} \times \left(1-s_{accept}\right) \right)  & \mathrm{if \; buying}  \\
        \end{array}\right. \\
        % その他変数の説明
        s_{accept} &= \left\{\begin{array}{ll}
            0.2 & \mathrm{if} \; r \geq 3 \\
            0   & \mathrm{if} \; r < 3    \\
        \end{array}\right. \\
        r &= \dfrac{p^{opp\_offer}_{i,j} - p^{opp\_offer}_{i,j-1}}{p^{opp\_offer}_{i,j-1} - p^{opp\_offer}_{i,j-2}} \\
        p^{best\_selling}_{i} &= \max\left(p^{opp\_offer}_{i,0}, p^{opp\_offer}_{i,1}, \dots, p^{opp\_offer}_{i,j}\right) \\
        p^{best\_buying}_{i} &= \min\left(p^{opp\_offer}_{i,0}, p^{opp\_offer}_{i,1}, \dots, p^{opp\_offer}_{i,j}\right) \\
    \end{split}
\end{equation*}
\begin{equation*}
    \begin{split}
        p^{best\_contract} &: \; \text{交渉相手との合意価格の中で自分にとって最も良いもの} \\
        p^{opp\_offer}_{i,j} &: \; \text{i日目jステップ目の相手のoffer価格}
    \end{split}
\end{equation*}

\section{\textrm{Risk Management}}
全く合意できないというリスクを対処するため,GentleはSelf factor($S$)をもとに譲歩したOfferを提案する.具体的には,次のような条件下で譲歩する.
\begin{enumerate}[\hspace{20mm}1\textrm{:}]
    \item \hspace{10mm} if $t^{sim}>0.3$ and $S<0.5$:
    \item \hspace{10mm} \qquad $s^{offer} = s^{offer} - 1$
\end{enumerate}
Self factorは,自身の交渉の進捗を表す指標であり,次のように求められる.
\begin{equation*}
    \begin{split}
        S &= \dfrac{2}{3}AR + \dfrac{1}{3}AP \\
        AR &= \left\{\begin{array}{ll}
            \dfrac{\text{number of success simulation steps}}{i} & \text{if Gentle has one or more agreements} \\
            1                                                    & \text{if Gentle has no agreement}           \\
        \end{array}\right. \\
        AP &= \left\{\begin{array}{ll}
            0.5 - b^{is\_selling} \times \dfrac{p^{prev\_contract}-TP}{TP} & \text{if Gentle has one or more agreements} \\
            0.5                                                            & \text{if Gentle has no agreement}           \\
        \end{array}\right. \\
    \end{split}
\end{equation*}
\begin{equation*}
    \begin{split}
        % AR &: \text{交渉開始から現在までの交渉成功割合} \\
        % AP &: \text{合意価格が良いかどうか} \\
        p^{prev\_contract} &: \text{直近の合意価格} \\
        TP &: \text{trading price} \\
    \end{split}
\end{equation*}

\section{\textrm{Evaluation}}
\end{document}
