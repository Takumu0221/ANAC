% sample.tex
\documentclass[uplatex, 10pt, a4j]{jsarticle}
\usepackage{authblk}
\usepackage{amsmath}
\usepackage{amssymb}
\usepackage{amsfonts}
\usepackage{latexsym}
\usepackage[top=20truemm,bottom=20truemm,left=15truemm,right=15truemm]{geometry}
\usepackage{enumerate}

\begin{document}
% 本文
\title{Gentle}
\author{Takumu Shimizu}
\affil{Tokyo University of Agriculture and Technology}
\affil{\vspace{-4mm}\normalsize\texttt{shimizu@katfuji.lab.tuat.ac.jp}}
\date{\vspace{-15mm}}

\maketitle

\section{\textrm{Negotiation Management}}
Gentle inherits \textit{AdaptiveAgent} and deals with opponents independently. 
Negotiation choices are the same way of the class, that is, Gentle continues to deal with all opponents until its needs $q^{need}$ are fulfilled.
$q^{need}$ is the quantity of exogenous contracts minus that of agreements for the day.
We don't set utility function and the decision of Gentle is mainly based on the unit price.
The following shows the trading information used as the basis for the decision.
\begin{equation*}
    \begin{split}
        i&:\text{simulation step}, \; j:\text{negotiation step}, \; t^{nego}:\text{negotiation time} \\
        n^{contract}_{offer}&:\text{number of agreements by Gentle's offer}, \; n^{contract}_{accept}: \; \text{number of agreements by Gentle's accept} \\
        p^{max}&:\text{maximum unit price of the negotiation}, \; p^{min}:\text{minimum unit price of the negotiation} \\
        p^{best\_contract} &: \; \text{the best unit price of the agreements with the opponent} \\
        p^{worst\_contract} &: \; \text{the worst unit price of the agreements with the opponent} \\
        p^{opp\_offer}_{i,j} &: \; \text{offer price of the opponent on day $i$, step $j$} \\
        p^{best\_selling}_{i} &= \max\left(p^{opp\_offer}_{i,0}, p^{opp\_offer}_{i,1}, \dots, p^{opp\_offer}_{i,j}\right) \\
        p^{best\_buying}_{i} &= \min\left(p^{opp\_offer}_{i,0}, p^{opp\_offer}_{i,1}, \dots, p^{opp\_offer}_{i,j}\right) \\    
    \end{split}
\end{equation*}

\subsection{\textrm{Offering Strategy}}
Gentle determines the offer unit price $p^{offer}$ and the offer quantity $q^{offer}$ based on some trading information.
The symbols $\prec$ and $\succeq$ are defined as the sign of inequalities that indicate whether the price is good or bad for Gentle.
In the cases, 5 $\prec$ 10 is true for the seller agent, and 5 $\succeq$ 10 is also true for the buyer agent.
$s^{offer}$ in the equation is a slack variable related to the offer, and the larger this value, the more bullish the offer.
We set a threshold $\tau$ that is the criterion for determining whether a given price is bad for Gentle or not.
\begin{equation*}
    \begin{split}
        % offer price
        p^{offer}&=\left\{\begin{array}{ll}
            \dfrac{p^{max}+p^{min}}{2} \times \left(1+type\times s_{offer}\right) & \mathrm{if} \; n^{contract}_{accept}+n^{contract}_{offer}=0                                      \\ [3mm]
            p^{worst\_contract}                                                              & \mathrm{if} \; n^{contract}_{accept} \geq 1 \; \mathrm{and} \; p^{worst\_contract} \succeq  \tau \\ [1mm]
                                                                                             & \mathrm{or} \; n^{contract}_{accept}<1 \; \mathrm{and} \; n^{contract}_{offer} \geq 1            \\ [3mm]
            p^{worst\_contract} \times \left(1 + type \times 0.1\right)           & \mathrm{if} \; n^{contract}_{accept} \geq 1 \; \mathrm{and} \; p^{worst\_contract} \prec \tau    \\
        \end{array}\right. \\
        q^{offer} &= q^{need} \\
    \end{split}
\end{equation*}
\begin{equation*}
    \begin{split}
        % その他変数の説明
        type&=\left\{\begin{array}{ll}
            1  & \mathrm{if \; selling} \\
            -1 & \mathrm{if \; buying}  \\
        \end{array}\right. \\
        s_{offer} &= 0.2-0.5 \times \min{\left(\dfrac{t^{nego}}{0.3}, 1\right)} \\
        \tau &= \left\{\begin{array}{ll}
            p^{max} - 19^{-0.2} \times \left(p^{max} - p^{min}\right) & \mathrm{if \; selling} \\
            p^{min} + 19^{-0.2} \times \left(p^{max} - p^{min}\right) & \mathrm{if \; buying}  \\
        \end{array}\right. \\
    \end{split}
\end{equation*}


\subsection{\textrm{Acceptance Strategy}}
Gentle determines the acceptable unit price $p^{offer}$ based on some trading information. The quantity is not taken into account.
$s^{accept}$ in the equation is a slack variable related to acceptance, and the larger this value, the more concessional acceptance.
We set a variable $r$ that is the ratio of the change in the opponent's concession rate.
\begin{equation*}
    \begin{split}
        % accept price
        p^{accept} &= \left\{\begin{array}{ll}
            \max \left(p^{best\_selling}_{i}, p^{best\_contract} \times \left(1+s_{accept}\right) \right) & \mathrm{if \; selling} \\ [3mm]
            \min \left(p^{best\_buying}_{i}, p^{best\_contract} \times \left(1-s_{accept}\right) \right)  & \mathrm{if \; buying}  \\
        \end{array}\right. \\
    \end{split}
\end{equation*}
\begin{equation*}
    \begin{split}
        % その他変数の説明
        s_{accept} &= \left\{\begin{array}{ll}
            0.2 & \mathrm{if} \; r \geq 3 \\
            0   & \mathrm{if} \; r < 3    \\
        \end{array}\right. \\
        r &= \dfrac{p^{opp\_offer}_{i,j} - p^{opp\_offer}_{i,j-1}}{p^{opp\_offer}_{i,j-1} - p^{opp\_offer}_{i,j-2}} \\
    \end{split}
\end{equation*}

\section{\textrm{Risk Management}}
To deal with the risk of not being able to agree at all, Gentle makes concessional offers based on the Self factor $S$.
Specifically, it makes concessions under the following conditions. Now, we set $t^sim$ that represents the simulation time.
\begin{enumerate}[\hspace{20mm}1\textrm{:}]
    \item \hspace{10mm} if $t^{sim}>0.3$ and $S<0.5$:
    \item \hspace{10mm} \qquad $s^{offer} = s^{offer} - 1$
\end{enumerate}
Self factor represents the progress of Gentle's own negotiation, and is caluculated as follows.
\begin{equation*}
    \begin{split}
        S &= \dfrac{2}{3}AR + \dfrac{1}{3}AP \\
        AR &= \left\{\begin{array}{ll}
            \dfrac{\text{number of success simulation steps}}{i} & \text{if Gentle has one or more agreements} \\
            1                                                    & \text{if Gentle has no agreement}           \\
        \end{array}\right. \\
        AP &= \left\{\begin{array}{ll}
            0.5 - type \times \dfrac{p^{prev\_contract}-TP}{TP} & \text{if Gentle has one or more agreements} \\
            0.5                                                            & \text{if Gentle has no agreement}           \\
        \end{array}\right. \\
    \end{split}
\end{equation*}
\begin{equation*}
    \begin{split}
        p^{prev\_contract} &: \text{most recently agreement price} \\
        TP &: \text{trading price} \\
    \end{split}
\end{equation*}

\section{\textrm{Evaluation}}
Gentleを \textit{LearningAgent} および \textit{AdaptiveAgent} を対戦相手としてシミュレーションを行った.その結果を表\ref{table:result}に示す.
\begin{table}[htbp]
    \caption{Gentleのシミュレーション結果}
    \label{table:result}
    \centering
    \begin{tabular}{ccccccc}
        \hline
        Agent                  & score    & min       & Q1       & median   & Q3       & max      \\
        \hline \hline
        Gentle                 & 1.060164 & 0.581421  & 0.964384 & 1.064991 & 1.130878 & 1.705256 \\
        \textit{LearningAgent} & 0.967132 & 0.553072  & 0.858785 & 0.958567 & 1.050838 & 1.708739 \\
        \textit{AdaptiveAgent} & 0.903169 & -0.072392 & 0.833647 & 0.971782 & 1.049998 & 1.146752 \\
        \hline
    \end{tabular}
\end{table}

結果から,Gentle はminからmaxの全てのスコアにおいて他のエージェントと同等かそれ以上となっている.よって,多くの場面で Gentle の交渉戦略が優れていると言える.
このような結果になった理由としては,Gentle は$p^{worst\_contract}$を考慮していることが挙げられる.
この$p^{worst\_contract}$は,交渉相手が受け入れるかどうかの基準として用いる価格であるため,これに近い価格でofferすることでより多くの合意を引き出すことができる.
\end{document}
